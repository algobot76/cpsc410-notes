\documentclass{article}
\usepackage[utf8]{inputenc}
\usepackage[a4paper, total={6in, 8in}]{geometry}

\title{CPSC 410 Notes}
\author{Kaitian Xie}
\date{June 2020}

\begin{document}

\maketitle
\pagebreak

\tableofcontents
\pagebreak

\section{Program Analysis and Visualization}

\subsection{Learning Objectives}

\begin{itemize}
    \item Differentiate the main types of program analysis, comparing their potential applications, strengths, and weaknesses.
    \item Design program analyses to extract and synthesise information to aid programmers with their daily work.
    \item Implement simple program analyses and visualizations of their results.
    \item Create useful visualizations of programming-related data.
    \item Justify the likely usefulness of program analyses and visualizations.
    \item Evaluate particular program analyses/visualizations using empirical studies.
    \item Categorise analyses according to the main types presented here, justifying which are necessary/useful for which tasks.
    \item Contrast the pros and cons of raw-data vs. static program analyses, in situations where either could be applicable.
    \item Describe the high-level implementation strategies for making large-scale static/meta-property analyses efficient.
    \item Justify which properties about code a value-agnostic static analysis can be used for (or cannot be used for).
    \item Provide examples of key main points for value-agnostic static analyses.
    \item Define the notion of all-executions properties, giving examples.
    \item Explain the key ideas of the Static Program Slicing technique presented here.
    \item Apply the Static Program Slicing technique to produce program slices for simple imperative programs.
    \item Explain the causes of approximation for static program analyses, giving examples.
    \item Contrast dynamic program analyses with their static counterparts, explaining the typical trade-offs.
    \item Contrast the pros and cons of the main implementation methods for dynamic program analyses (when instrumentation is performed).
    \item Apply the Dynamic Program Slicing technique to produce program slices for simple imperative programs.
    \item Explain the concepts of over- and under-approximation, and their practical consequences for the analysis of correctness properties.
    \item Derive appropriate Failure Conditions for given statements and correctness properties of interest.
    \item Define the key ideas of the symbolic execution techniques presented here, and their meanings in the context of checking correctness properties.
    \item Apply the symbolic execution technique to simple imperative programs.
    \item Justify which of the standard pain points for a program analysis are addressed by symbolic execution (and which are not).
\end{itemize}

\section{Digesting and Evaluating PL/SE Research Papers}

\subsection{Learning Objectives}

\begin{itemize}
    \item Explain the key things to look for when skim-reading a research paper.
    \item Apply your knowledge of empirical evaluation techniques to critically evaluate the evidence provided in a research paper.
    \item Create short, high-level, summaries of the main points of a research paper.
\end{itemize}

\end{document}
